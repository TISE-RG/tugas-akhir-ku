% tesis.tex
% =========
\documentclass[11pt, a4paper, onecolumn, oneside, final]{book}

% Ganti ini sesuai style yang dipakai:
% - kalau tetap ingin pakai yang lama: \input{eb-itb-thesis.sty}
% - kalau sudah di-rename:            \input{tise-itb-thesis.sty}
\input{tise-itb-thesis.sty}
\makeatletter
\makeatother

% --- Parameter tunggal ---
% thesis-params.tex
% =================
% Satu-satunya file yang perlu diubah mahasiswa

% --- Jenis naskah & program ---
\newcommand{\DocType}{TUGAS AKHIR}  % TUGAS AKHIR / TESIS / DISERTASI
\newcommand{\DegreeLong}{Sarjana Teknik}  % Magister Teknik / Doktor, dll.
\newcommand{\DegreeShort}{S.T.}           % M.T. / Dr. / Ph.D., dll.

\newcommand{\Program}{Program Studi Teknik Biomedis}
\newcommand{\Faculty}{Sekolah Teknik Elektro dan Informatika}
\newcommand{\University}{Institut Teknologi Bandung}

% --- Identitas penulis ---
\newcommand{\AuthorName}{Nama Lengkap Mahasiswa}
\newcommand{\AuthorNIM}{1234567890}

% --- Judul (boleh multi-baris) ---
\newcommand{\ThesisTitle}{%
  Judul Tugas Akhir / Tesis / Disertasi
  Yang Bisa Panjang dan Lebih dari Satu Baris
}

% --- Pembimbing ---
\newcommand{\AdvisorOneName}{Dr. Nama Pembimbing I}
\newcommand{\AdvisorOneNIP}{NIP. 123 456 789}

\newcommand{\AdvisorTwoName}{Ir. Nama Pembimbing II, M.T.}
\newcommand{\AdvisorTwoNIP}{NIP. 987 654 321}

% Jika hanya satu pembimbing, kosongkan saja AdvisorTwoName/NIP
% atau atur logika di file approval.

% --- Info lokasi & waktu ---
\newcommand{\City}{Bandung}
\newcommand{\MonthYear}{JULI 2026}
   % <--- semua identitas diambil dari sini

\begin{document}

% --- Konfigurasi dasar judul/penulis ---
\title{\ThesisTitle}
\author{%
  \AuthorName\\
  NIM : \AuthorNIM
}
\date{}  % biasanya kosong

% ================= FRONT MATTER =================
\frontmatter

% Cover dan lembar pengesahan (pakai parameter)
\input{frontmatter/000-cover}
\input{frontmatter/001-approval}

% Di bawah ini contoh lanjutan (opsional) kalau mau meniru if/eb-itb:
\input{frontmatter/002-statement}
\pagestyle{plain}
\input{frontmatter/003-abstract-id}
\input{frontmatter/004-abstract-en}
\input{frontmatter/005-forewords}
\tableofcontents
\addcontentsline{toc}{chapter}{DAFTAR ISI}
\listoffigures
\addcontentsline{toc}{chapter}{DAFTAR GAMBAR}
\listoftables
\addcontentsline{toc}{chapter}{DAFTAR TABEL}

% ================= MAIN MATTER ==================
\mainmatter


\include{chapters/01-pendahuluan}
\include{chapters/02-tinjauan-pustaka}
\include{chapters/03-metodologi}
\include{chapters/04-hasil-pembahasan}
\include{chapters/05-kesimpulan-saran}

% ================= BACK MATTER ==================
\appendix
\include{chapters/A-lampiran1}
\include{chapters/B-lampiran2}

\bibliographystyle{IEEEtran} % atau apa pun yang sesuai
\bibliography{references}    % references.bib

\end{document}
