% ==========================================================
% FILE: layout/frontmatter.tex
% DESKRIPSI: Mengatur tampilan bagian depan dokumen
% ==========================================================

% --- 1. COVER PAGE ---
\begin{titlepage}
    \centering
    \vspace*{1cm}
    {\Large \bfseries \MakeUppercase{\tipeDokumen}}\\[1.5cm]
    
    {\Large \bfseries \judulID}\\[0.5cm]
    {\itshape \judulEN}\\[2cm]
    
    \includegraphics[width=3.5cm]{images/cover-ganesha.jpg}\\[2cm]
    
    {\large Oleh:}\\[0.5cm]
    {\large \bfseries \namaPenulis}\\[0.2cm]
    {\large NIM: \nim}\\[2cm]
    
    {\large \MakeUppercase{\programStudi}}\\[0.2cm]
    {\large \MakeUppercase{\fakultas}}\\[0.2cm]
    {\large \MakeUppercase{\institusi}}\\[0.2cm]
    {\large \tahun}
\end{titlepage}

% --- 2. LEMBAR PENGESAHAN ---
\newpage
\thispagestyle{empty}
\begin{center}
    {\Large \bfseries LEMBAR PENGESAHAN}\\[1cm]
    {\bfseries \judulID}\\[1cm]
    
    Oleh:\\[0.5cm]
    {\bfseries \namaPenulis}\\
    NIM: \nim\\[1.5cm]
    
    Menyetujui,\\[0.5cm]
    Tim Pembimbing\\[1cm]
    
    \begin{tabular}{cc}
        \multicolumn{2}{c}{Bandung, \tanggalSidang} \\[2cm]
        \underline{\pembimbingSatu} & \underline{\pembimbingDua} \\
        \nipPembimbingSatu & \nipPembimbingDua \\
    \end{tabular}
\end{center}
\addcontentsline{toc}{chapter}{Lembar Pengesahan}

% --- 3. ABSTRAK (INDONESIA) ---
\newpage
\chapter*{Abstrak}
\addcontentsline{toc}{chapter}{Abstrak}
\begin{center}
    \textbf{\judulID}
\end{center}
\vspace{0.5cm}
Penelitian ini bertujuan untuk mengembangkan sistem kendali jarak
jauh\ldots{}

Hasil pengujian menunjukkan bahwa \textbf{akurasi sistem mencapai 98\%}.
Metode yang digunakan berbasis pada \emph{Artificial Intelligence}
dengan algoritma CNN.

% --- 4. ABSTRACT (ENGLISH) ---
\newpage
\chapter*{Abstract}
\addcontentsline{toc}{chapter}{Abstract}
\begin{center}
    \textit{\textbf{\judulEN}}
\end{center}
\vspace{0.5cm}
\input{content/abstrak-en}

% --- 5. KATA PENGANTAR ---
\newpage
\chapter*{Kata Pengantar}
\addcontentsline{toc}{chapter}{Kata Pengantar}
Puji syukur kepada Tuhan YME\ldots{} Saya ingin berterima kasih
kepada\ldots{}

Penelitian ini bertujuan untuk mengembangkan sistem kendali jarak
jauh\ldots{}

Hasil pengujian menunjukkan bahwa \textbf{akurasi sistem mencapai 98\%}.
Metode yang digunakan berbasis pada \emph{Artificial Intelligence}
dengan algoritma CNN.

% --- 6. DAFTAR ISI ---
\newpage
\tableofcontents
\listoffigures
\listoftables   