% Options for packages loaded elsewhere
\PassOptionsToPackage{unicode}{hyperref}
\PassOptionsToPackage{hyphens}{url}
\PassOptionsToPackage{dvipsnames,svgnames,x11names}{xcolor}
%
\documentclass[
  letterpaper,
]{article}

\usepackage{amsmath,amssymb}
\usepackage{iftex}
\ifPDFTeX
  \usepackage[T1]{fontenc}
  \usepackage[utf8]{inputenc}
  \usepackage{textcomp} % provide euro and other symbols
\else % if luatex or xetex
  \usepackage{unicode-math}
  \defaultfontfeatures{Scale=MatchLowercase}
  \defaultfontfeatures[\rmfamily]{Ligatures=TeX,Scale=1}
\fi
\usepackage{lmodern}
\ifPDFTeX\else  
    % xetex/luatex font selection
\fi
% Use upquote if available, for straight quotes in verbatim environments
\IfFileExists{upquote.sty}{\usepackage{upquote}}{}
\IfFileExists{microtype.sty}{% use microtype if available
  \usepackage[]{microtype}
  \UseMicrotypeSet[protrusion]{basicmath} % disable protrusion for tt fonts
}{}
\makeatletter
\@ifundefined{KOMAClassName}{% if non-KOMA class
  \IfFileExists{parskip.sty}{%
    \usepackage{parskip}
  }{% else
    \setlength{\parindent}{0pt}
    \setlength{\parskip}{6pt plus 2pt minus 1pt}}
}{% if KOMA class
  \KOMAoptions{parskip=half}}
\makeatother
\usepackage{xcolor}
\setlength{\emergencystretch}{3em} % prevent overfull lines
\setcounter{secnumdepth}{5}
% Make \paragraph and \subparagraph free-standing
\makeatletter
\ifx\paragraph\undefined\else
  \let\oldparagraph\paragraph
  \renewcommand{\paragraph}{
    \@ifstar
      \xxxParagraphStar
      \xxxParagraphNoStar
  }
  \newcommand{\xxxParagraphStar}[1]{\oldparagraph*{#1}\mbox{}}
  \newcommand{\xxxParagraphNoStar}[1]{\oldparagraph{#1}\mbox{}}
\fi
\ifx\subparagraph\undefined\else
  \let\oldsubparagraph\subparagraph
  \renewcommand{\subparagraph}{
    \@ifstar
      \xxxSubParagraphStar
      \xxxSubParagraphNoStar
  }
  \newcommand{\xxxSubParagraphStar}[1]{\oldsubparagraph*{#1}\mbox{}}
  \newcommand{\xxxSubParagraphNoStar}[1]{\oldsubparagraph{#1}\mbox{}}
\fi
\makeatother

\usepackage{color}
\usepackage{fancyvrb}
\newcommand{\VerbBar}{|}
\newcommand{\VERB}{\Verb[commandchars=\\\{\}]}
\DefineVerbatimEnvironment{Highlighting}{Verbatim}{commandchars=\\\{\}}
% Add ',fontsize=\small' for more characters per line
\usepackage{framed}
\definecolor{shadecolor}{RGB}{241,243,245}
\newenvironment{Shaded}{\begin{snugshade}}{\end{snugshade}}
\newcommand{\AlertTok}[1]{\textcolor[rgb]{0.68,0.00,0.00}{#1}}
\newcommand{\AnnotationTok}[1]{\textcolor[rgb]{0.37,0.37,0.37}{#1}}
\newcommand{\AttributeTok}[1]{\textcolor[rgb]{0.40,0.45,0.13}{#1}}
\newcommand{\BaseNTok}[1]{\textcolor[rgb]{0.68,0.00,0.00}{#1}}
\newcommand{\BuiltInTok}[1]{\textcolor[rgb]{0.00,0.23,0.31}{#1}}
\newcommand{\CharTok}[1]{\textcolor[rgb]{0.13,0.47,0.30}{#1}}
\newcommand{\CommentTok}[1]{\textcolor[rgb]{0.37,0.37,0.37}{#1}}
\newcommand{\CommentVarTok}[1]{\textcolor[rgb]{0.37,0.37,0.37}{\textit{#1}}}
\newcommand{\ConstantTok}[1]{\textcolor[rgb]{0.56,0.35,0.01}{#1}}
\newcommand{\ControlFlowTok}[1]{\textcolor[rgb]{0.00,0.23,0.31}{\textbf{#1}}}
\newcommand{\DataTypeTok}[1]{\textcolor[rgb]{0.68,0.00,0.00}{#1}}
\newcommand{\DecValTok}[1]{\textcolor[rgb]{0.68,0.00,0.00}{#1}}
\newcommand{\DocumentationTok}[1]{\textcolor[rgb]{0.37,0.37,0.37}{\textit{#1}}}
\newcommand{\ErrorTok}[1]{\textcolor[rgb]{0.68,0.00,0.00}{#1}}
\newcommand{\ExtensionTok}[1]{\textcolor[rgb]{0.00,0.23,0.31}{#1}}
\newcommand{\FloatTok}[1]{\textcolor[rgb]{0.68,0.00,0.00}{#1}}
\newcommand{\FunctionTok}[1]{\textcolor[rgb]{0.28,0.35,0.67}{#1}}
\newcommand{\ImportTok}[1]{\textcolor[rgb]{0.00,0.46,0.62}{#1}}
\newcommand{\InformationTok}[1]{\textcolor[rgb]{0.37,0.37,0.37}{#1}}
\newcommand{\KeywordTok}[1]{\textcolor[rgb]{0.00,0.23,0.31}{\textbf{#1}}}
\newcommand{\NormalTok}[1]{\textcolor[rgb]{0.00,0.23,0.31}{#1}}
\newcommand{\OperatorTok}[1]{\textcolor[rgb]{0.37,0.37,0.37}{#1}}
\newcommand{\OtherTok}[1]{\textcolor[rgb]{0.00,0.23,0.31}{#1}}
\newcommand{\PreprocessorTok}[1]{\textcolor[rgb]{0.68,0.00,0.00}{#1}}
\newcommand{\RegionMarkerTok}[1]{\textcolor[rgb]{0.00,0.23,0.31}{#1}}
\newcommand{\SpecialCharTok}[1]{\textcolor[rgb]{0.37,0.37,0.37}{#1}}
\newcommand{\SpecialStringTok}[1]{\textcolor[rgb]{0.13,0.47,0.30}{#1}}
\newcommand{\StringTok}[1]{\textcolor[rgb]{0.13,0.47,0.30}{#1}}
\newcommand{\VariableTok}[1]{\textcolor[rgb]{0.07,0.07,0.07}{#1}}
\newcommand{\VerbatimStringTok}[1]{\textcolor[rgb]{0.13,0.47,0.30}{#1}}
\newcommand{\WarningTok}[1]{\textcolor[rgb]{0.37,0.37,0.37}{\textit{#1}}}

\providecommand{\tightlist}{%
  \setlength{\itemsep}{0pt}\setlength{\parskip}{0pt}}\usepackage{longtable,booktabs,array}
\usepackage{calc} % for calculating minipage widths
% Correct order of tables after \paragraph or \subparagraph
\usepackage{etoolbox}
\makeatletter
\patchcmd\longtable{\par}{\if@noskipsec\mbox{}\fi\par}{}{}
\makeatother
% Allow footnotes in longtable head/foot
\IfFileExists{footnotehyper.sty}{\usepackage{footnotehyper}}{\usepackage{footnote}}
\makesavenoteenv{longtable}
\usepackage{graphicx}
\makeatletter
\def\maxwidth{\ifdim\Gin@nat@width>\linewidth\linewidth\else\Gin@nat@width\fi}
\def\maxheight{\ifdim\Gin@nat@height>\textheight\textheight\else\Gin@nat@height\fi}
\makeatother
% Scale images if necessary, so that they will not overflow the page
% margins by default, and it is still possible to overwrite the defaults
% using explicit options in \includegraphics[width, height, ...]{}
\setkeys{Gin}{width=\maxwidth,height=\maxheight,keepaspectratio}
% Set default figure placement to htbp
\makeatletter
\def\fps@figure{htbp}
\makeatother
% definitions for citeproc citations
\NewDocumentCommand\citeproctext{}{}
\NewDocumentCommand\citeproc{mm}{%
  \begingroup\def\citeproctext{#2}\cite{#1}\endgroup}
\makeatletter
 % allow citations to break across lines
 \let\@cite@ofmt\@firstofone
 % avoid brackets around text for \cite:
 \def\@biblabel#1{}
 \def\@cite#1#2{{#1\if@tempswa , #2\fi}}
\makeatother
\newlength{\cslhangindent}
\setlength{\cslhangindent}{1.5em}
\newlength{\csllabelwidth}
\setlength{\csllabelwidth}{3em}
\newenvironment{CSLReferences}[2] % #1 hanging-indent, #2 entry-spacing
 {\begin{list}{}{%
  \setlength{\itemindent}{0pt}
  \setlength{\leftmargin}{0pt}
  \setlength{\parsep}{0pt}
  % turn on hanging indent if param 1 is 1
  \ifodd #1
   \setlength{\leftmargin}{\cslhangindent}
   \setlength{\itemindent}{-1\cslhangindent}
  \fi
  % set entry spacing
  \setlength{\itemsep}{#2\baselineskip}}}
 {\end{list}}
\usepackage{calc}
\newcommand{\CSLBlock}[1]{\hfill\break\parbox[t]{\linewidth}{\strut\ignorespaces#1\strut}}
\newcommand{\CSLLeftMargin}[1]{\parbox[t]{\csllabelwidth}{\strut#1\strut}}
\newcommand{\CSLRightInline}[1]{\parbox[t]{\linewidth - \csllabelwidth}{\strut#1\strut}}
\newcommand{\CSLIndent}[1]{\hspace{\cslhangindent}#1}

\makeatletter
\@ifpackageloaded{bookmark}{}{\usepackage{bookmark}}
\makeatother
\makeatletter
\@ifpackageloaded{caption}{}{\usepackage{caption}}
\AtBeginDocument{%
\ifdefined\contentsname
  \renewcommand*\contentsname{Table of contents}
\else
  \newcommand\contentsname{Table of contents}
\fi
\ifdefined\listfigurename
  \renewcommand*\listfigurename{List of Figures}
\else
  \newcommand\listfigurename{List of Figures}
\fi
\ifdefined\listtablename
  \renewcommand*\listtablename{List of Tables}
\else
  \newcommand\listtablename{List of Tables}
\fi
\ifdefined\figurename
  \renewcommand*\figurename{Figure}
\else
  \newcommand\figurename{Figure}
\fi
\ifdefined\tablename
  \renewcommand*\tablename{Table}
\else
  \newcommand\tablename{Table}
\fi
}
\@ifpackageloaded{float}{}{\usepackage{float}}
\floatstyle{ruled}
\@ifundefined{c@chapter}{\newfloat{codelisting}{h}{lop}}{\newfloat{codelisting}{h}{lop}[chapter]}
\floatname{codelisting}{Listing}
\newcommand*\listoflistings{\listof{codelisting}{List of Listings}}
\makeatother
\makeatletter
\makeatother
\makeatletter
\@ifpackageloaded{caption}{}{\usepackage{caption}}
\@ifpackageloaded{subcaption}{}{\usepackage{subcaption}}
\makeatother

\ifLuaTeX
  \usepackage{selnolig}  % disable illegal ligatures
\fi
\usepackage{bookmark}

\IfFileExists{xurl.sty}{\usepackage{xurl}}{} % add URL line breaks if available
\urlstyle{same} % disable monospaced font for URLs
\hypersetup{
  pdftitle={Manuskrop},
  pdfauthor={Norah Jones},
  colorlinks=true,
  linkcolor={blue},
  filecolor={Maroon},
  citecolor={Blue},
  urlcolor={Blue},
  pdfcreator={LaTeX via pandoc}}


\title{Manuskrop}
\author{Norah Jones}
\date{2025-01-12}

\begin{document}
\maketitle

\renewcommand*\contentsname{Table of contents}
{
\hypersetup{linkcolor=}
\setcounter{tocdepth}{2}
\tableofcontents
}

\bookmarksetup{startatroot}

\chapter{An Essay with Integrated Code, Diagrams, Tables, Equations, and
Images}\label{an-essay-with-integrated-code-diagrams-tables-equations-and-images}

\bookmarksetup{startatroot}

\chapter{Introduction}\label{introduction}

Modern scientific and engineering communication benefits from a
\textbf{blend of media}:\\
plots produced by code, structural diagrams, numerical tables,
mathematical equations, and illustrative images.

In this short essay we demonstrate how a Quarto document can integrate:

\begin{itemize}
\tightlist
\item
  a Python-generated plot (Figure~\ref{fig-sine}),
\item
  a Mermaid diagram (Figure~\ref{fig-flow}),
\item
  a data table (Table~\ref{tbl-stats}),
\item
  a mathematical equation with a label (Equation~\ref{eq-gauss}),
\item
  and an external picture (Figure~\ref{fig-photo}).
\end{itemize}

Each element is encapsulated in its own block, given a caption, and
cross-referenced in the text.

\bookmarksetup{startatroot}

\chapter{1. Python Plots for Quantitative
Insight}\label{python-plots-for-quantitative-insight}

Numerical phenomena are often best understood visually.\\
In Figure~\ref{fig-sine}, we use a Python code block to generate a
simple plot of a sine wave and its noisy variant.\\
This kind of plot is typical in courses on signals and systems,
time-series analysis, and scientific computing.

\begin{Shaded}
\begin{Highlighting}[]
\ImportTok{import}\NormalTok{ numpy }\ImportTok{as}\NormalTok{ np}
\ImportTok{import}\NormalTok{ matplotlib.pyplot }\ImportTok{as}\NormalTok{ plt}

\CommentTok{\# Generate data}
\NormalTok{t }\OperatorTok{=}\NormalTok{ np.linspace(}\DecValTok{0}\NormalTok{, }\DecValTok{2}\OperatorTok{*}\NormalTok{np.pi, }\DecValTok{200}\NormalTok{)}
\NormalTok{x }\OperatorTok{=}\NormalTok{ np.sin(t)}
\NormalTok{noise }\OperatorTok{=} \FloatTok{0.2} \OperatorTok{*}\NormalTok{ np.random.randn(}\BuiltInTok{len}\NormalTok{(t))}
\NormalTok{y }\OperatorTok{=}\NormalTok{ x }\OperatorTok{+}\NormalTok{ noise}

\CommentTok{\# Create plot}
\NormalTok{plt.figure()}
\NormalTok{plt.plot(t, x, label}\OperatorTok{=}\StringTok{"Clean sine x(t)"}\NormalTok{)}
\NormalTok{plt.plot(t, y, }\StringTok{"."}\NormalTok{, label}\OperatorTok{=}\StringTok{"Noisy samples y(t)"}\NormalTok{)}
\NormalTok{plt.xlabel(}\StringTok{"t (radians)"}\NormalTok{)}
\NormalTok{plt.ylabel(}\StringTok{"Amplitude"}\NormalTok{)}
\NormalTok{plt.legend()}
\NormalTok{plt.tight\_layout()}
\end{Highlighting}
\end{Shaded}

\begin{figure}[H]

\centering{

\includegraphics{index_files/figure-pdf/fig-sine-output-1.pdf}

}

\caption{\label{fig-sine}Sine wave and noisy observation generated using
Python and Matplotlib.}

\end{figure}%

The code in Figure~\ref{fig-sine} not only produces a figure but also
documents the process. Because it lives in the same document as the
explanation, it strongly supports reproducible research.

\bookmarksetup{startatroot}

\chapter{2. Mermaid Diagrams for Conceptual
Structure}\label{mermaid-diagrams-for-conceptual-structure}

While Python plots excel at quantitative views, \textbf{Mermaid
diagrams} provide concise visualizations of \textbf{conceptual flows and
system structure}.

In Figure~\ref{fig-flow} we use a Mermaid flowchart to represent a
simple data-processing pipeline, echoing the kind of workflow that might
use the data generated in Figure~\ref{fig-sine}.

\begin{figure}

\centering{

\begin{Shaded}
\begin{Highlighting}[]
\NormalTok{flowchart LR}
\NormalTok{    A[Raw Data] {-}{-}\textgreater{} B[Preprocessing]}
\NormalTok{    B {-}{-}\textgreater{} C[Modeling]}
\NormalTok{    C {-}{-}\textgreater{} D[Evaluation]}
\NormalTok{    D {-}{-}\textgreater{} E[Report \& Visualization]}
\end{Highlighting}
\end{Shaded}

\includegraphics[width=7.92in,height=0.51in]{index_files/figure-latex/mermaid-figure-1.png}

}

\caption{\label{fig-flow}Mermaid flowchart illustrating a simple data
processing pipeline.}

\end{figure}%

The structure in Figure~\ref{fig-flow} complements the numerical view in
Figure~\ref{fig-sine}: one shows \emph{what the data look like}, the
other shows \emph{what we do with the data}.

\bookmarksetup{startatroot}

\chapter{3. Tables for Compact Numerical
Summaries}\label{tables-for-compact-numerical-summaries}

Sometimes a \textbf{compact tabular summary} is more useful than a
figure. In Table~\ref{tbl-stats} we compute basic statistics from the
same sine--noise data used in Figure~\ref{fig-sine}. Here we rely on
Python again, but this time the result is rendered as a table instead of
a plot.

\begin{Shaded}
\begin{Highlighting}[]
\ImportTok{import}\NormalTok{ pandas }\ImportTok{as}\NormalTok{ pd}
\ImportTok{import}\NormalTok{ numpy }\ImportTok{as}\NormalTok{ np}

\NormalTok{stats }\OperatorTok{=}\NormalTok{ \{}
    \StringTok{"series"}\NormalTok{: [}\StringTok{"x(t) clean"}\NormalTok{, }\StringTok{"y(t) noisy"}\NormalTok{],}
    \StringTok{"mean"}\NormalTok{: [np.mean(x), np.mean(y)],}
    \StringTok{"std"}\NormalTok{: [np.std(x), np.std(y)],}
    \StringTok{"min"}\NormalTok{: [np.}\BuiltInTok{min}\NormalTok{(x), np.}\BuiltInTok{min}\NormalTok{(y)],}
    \StringTok{"max"}\NormalTok{: [np.}\BuiltInTok{max}\NormalTok{(x), np.}\BuiltInTok{max}\NormalTok{(y)]}
\NormalTok{\}}

\NormalTok{df\_stats }\OperatorTok{=}\NormalTok{ pd.DataFrame(stats)}
\NormalTok{df\_stats}
\end{Highlighting}
\end{Shaded}

\begin{longtable}[]{@{}llllll@{}}

\caption{\label{tbl-stats}Summary statistics for the clean signal x(t)
and noisy observation y(t).}

\tabularnewline

\toprule\noalign{}
& series & mean & std & min & max \\
\midrule\noalign{}
\endhead
\bottomrule\noalign{}
\endlastfoot
0 & x(t) clean & 0.000000 & 0.705337 & -0.999969 & 0.999969 \\
1 & y(t) noisy & -0.014908 & 0.749645 & -1.340742 & 1.365809 \\

\end{longtable}

The table Table~\ref{tbl-stats} complements Figure~\ref{fig-sine} by
giving numerical values for mean, standard deviation, and range.
Together they allow the reader to both \textbf{see} and \textbf{measure}
the effect of noise on the signal.

\bookmarksetup{startatroot}

\chapter{4. Equations for Analytical
Foundations}\label{equations-for-analytical-foundations}

Behind the numerical experiments in Figure~\ref{fig-sine} and the
workflow in Figure~\ref{fig-flow} lies a rich \textbf{analytical
foundation}. Equations provide a precise language for expressing models
and assumptions.

For example, the well-known \textbf{Gaussian probability density
function} can be written as:

\begin{equation}\phantomsection\label{eq-gauss}{
f(x) = \frac{1}{\sqrt{2\pi\sigma^2}}
\exp!\left(
-,\frac{(x - \mu)^2}{2\sigma^2}
\right)
}\end{equation}

Equation Equation~\ref{eq-gauss} is frequently used to model noise such
as the perturbations added to the sine wave in Figure~\ref{fig-sine}. By
referring directly to Equation~\ref{eq-gauss} in the text, we connect
the \textbf{mathematical model}, the \textbf{simulation code}, and the
\textbf{visual outputs} in a single narrative.

\bookmarksetup{startatroot}

\chapter{5. Images for Context and
Intuition}\label{images-for-context-and-intuition}

Finally, \textbf{photographs and other images} can add context,
intuition, or aesthetic value that diagrams and equations alone cannot
provide. Consider Figure~\ref{fig-photo}, which you might imagine as an
illustration of a data-collection setting (for instance, a sensor
observing a natural environment).

\begin{quote}
\textbf{Note:} Replace the file path below
(\texttt{images/mountains.jpg}) with any real image in your project
directory.
\end{quote}

\begin{figure}

\centering{

\includegraphics{images/mountains.jpg}

}

\caption{\label{fig-photo}Illustrative photograph suggesting a
measurement environment, e.g., sensors observing a physical system.}

\end{figure}%

Figure Figure~\ref{fig-photo} can be used to anchor the abstract
discussion of signals, systems, and models in a concrete, physical
setting. The same document thus ties together \textbf{real-world
context} (Figure~\ref{fig-photo}), \textbf{system structure}
(Figure~\ref{fig-flow}), \textbf{data behavior} (Figure~\ref{fig-sine}),
\textbf{numerical summaries} (Table~\ref{tbl-stats}), and
\textbf{analytical formulas} (Equation~\ref{eq-gauss}).

\bookmarksetup{startatroot}

\chapter{Conclusion}\label{conclusion}

In this essay we have demonstrated how a single Quarto document can
blend:

\begin{itemize}
\tightlist
\item
  \textbf{Python plots} for numerical intuition (Figure~\ref{fig-sine}),
\item
  \textbf{Mermaid diagrams} for conceptual workflows
  (Figure~\ref{fig-flow}),
\item
  \textbf{Tables} for compact quantitative summaries
  (Table~\ref{tbl-stats}),
\item
  \textbf{Equations} for mathematical rigor (Equation~\ref{eq-gauss}),
\item
  \textbf{Images} for real-world context (Figure~\ref{fig-photo}).
\end{itemize}

By encapsulating each element in a block with a caption and
cross-referencing it from the text, Quarto encourages \textbf{coherent,
literate, and reproducible} technical writing.

The same pattern can be extended to larger projects: lecture notes,
research articles, technical reports, or even books---always keeping
\textbf{code, narrative, and visuals} in one unified, executable
manuscript.

\textasciitilde\textasciitilde\textasciitilde\textasciitilde{}

\bookmarksetup{startatroot}

\chapter*{Preface}\label{preface}
\addcontentsline{toc}{chapter}{Preface}

\markboth{Preface}{Preface}

This is a Quarto book.

To learn more about Quarto books visit
\url{https://quarto.org/docs/books}.

\bookmarksetup{startatroot}

\chapter{Introduction}\label{introduction-1}

This is a book created from markdown and executable code.

See Knuth (1984) for additional discussion of literate programming.

\bookmarksetup{startatroot}

\chapter{Ten Best Graphviz Diagram
Examples}\label{ten-best-graphviz-diagram-examples}

\bookmarksetup{startatroot}

\chapter{Introduction}\label{introduction-2}

Graphviz provides exceptional power for expressing structures, flows,
and relationships.\\
This essay demonstrates ten of the most useful Graphviz diagram types,
each shown in figures such as the simple block diagram in
Figure~\ref{fig-block} and the hierarchical tree in \textbf{?@fig-tree}.

Throughout the text, we reference diagrams using Quarto's
cross-referencing syntax (e.g., \textbf{?@fig-network},
\textbf{?@fig-state}), enabling clean academic writing.

\begin{center}\rule{0.5\linewidth}{0.5pt}\end{center}

\bookmarksetup{startatroot}

\chapter{1. Simple Signal--System Block
Diagram}\label{simple-signalsystem-block-diagram}

A minimal left-to-right block diagram is shown in
Figure~\ref{fig-block}.

\begin{figure}

\centering{

\includegraphics[width=5.5in,height=3.5in]{summary_files/figure-latex/dot-figure-1.png}

}

\caption{\label{fig-block}Simple Signal--System Block Diagram}

\end{figure}%

\begin{center}\rule{0.5\linewidth}{0.5pt}\end{center}

\bookmarksetup{startatroot}

\chapter{3. Directed Workflow Diagram}\label{directed-workflow-diagram}

The pipeline workflow in \textbf{?@fig-workflow} shows a multi-stage
process.

\includegraphics[width=5.5in,height=3.5in]{summary_files/figure-latex/dot-figure-9.png}

\begin{center}\rule{0.5\linewidth}{0.5pt}\end{center}

\bookmarksetup{startatroot}

\chapter{4. Decision Diagram}\label{decision-diagram}

A branching decision structure is presented in \textbf{?@fig-decision}.

\includegraphics[width=5.5in,height=3.5in]{summary_files/figure-latex/dot-figure-8.png}

\begin{center}\rule{0.5\linewidth}{0.5pt}\end{center}

\bookmarksetup{startatroot}

\chapter{5. Undirected Network Graph}\label{undirected-network-graph}

Graphviz can model networks such as the undirected cluster in
\textbf{?@fig-network}.

\includegraphics[width=5.5in,height=3.5in]{summary_files/figure-latex/dot-figure-7.png}

\begin{center}\rule{0.5\linewidth}{0.5pt}\end{center}

\bookmarksetup{startatroot}

\chapter{6. State Machine Diagram}\label{state-machine-diagram}

The state transition model in \textbf{?@fig-state} illustrates system
behavior.

\includegraphics[width=5.5in,height=3.5in]{summary_files/figure-latex/dot-figure-6.png}

\begin{center}\rule{0.5\linewidth}{0.5pt}\end{center}

\bookmarksetup{startatroot}

\chapter{7. Clustered Graph (Subsystem
Grouping)}\label{clustered-graph-subsystem-grouping}

Clustering helps show subsystems; see \textbf{?@fig-cluster}.

\includegraphics[width=5.5in,height=3.5in]{summary_files/figure-latex/dot-figure-5.png}

\begin{center}\rule{0.5\linewidth}{0.5pt}\end{center}

\bookmarksetup{startatroot}

\chapter{8. Bipartite Graph}\label{bipartite-graph}

A classic bipartite structure is illustrated in
\textbf{?@fig-bipartite}.

\includegraphics[width=5.5in,height=3.5in]{summary_files/figure-latex/dot-figure-4.png}

\begin{center}\rule{0.5\linewidth}{0.5pt}\end{center}

\bookmarksetup{startatroot}

\chapter{9. DAG (Directed Acyclic
Graph)}\label{dag-directed-acyclic-graph}

The DAG in \textbf{?@fig-dag} is useful for tasks, scheduling, or
dependencies.

\includegraphics[width=5.5in,height=3.5in]{summary_files/figure-latex/dot-figure-3.png}

\begin{center}\rule{0.5\linewidth}{0.5pt}\end{center}

\bookmarksetup{startatroot}

\chapter{10. Flow Network with Edge
Capacities}\label{flow-network-with-edge-capacities}

Flow networks visualize constraints; the example is shown in
Figure~\ref{fig-flow}.

\includegraphics[width=5.5in,height=3.5in]{summary_files/figure-latex/dot-figure-2.png}

\begin{center}\rule{0.5\linewidth}{0.5pt}\end{center}

\bookmarksetup{startatroot}

\chapter{Conclusion}\label{conclusion-1}

These ten Graphviz examples---ranging from basic block diagrams (e.g.,
Figure~\ref{fig-block}) to sophisticated cluster graphs
(\textbf{?@fig-cluster}) and dependency DAGs (\textbf{?@fig-dag})---show
how expressive DOT is when embedded in Quarto.

Graphviz offers clarity, precision, and automation that make it
invaluable for engineering, system design, data modeling, and conceptual
explanation.

\bookmarksetup{startatroot}

\chapter{Ten PlantUML Diagram Examples in
Quarto}\label{ten-plantuml-diagram-examples-in-quarto}

\bookmarksetup{startatroot}

\chapter{Introduction}\label{introduction-3}

PlantUML is a powerful, text-based diagramming tool that integrates well
with Quarto.\\
With a few lines of code, we can include rich diagrams such as
\textbf{sequence}, \textbf{use case}, \textbf{class}, \textbf{activity},
\textbf{state}, \textbf{component}, \textbf{deployment},
\textbf{object}, \textbf{timing}, and \textbf{package} diagrams.

In this short essay, we illustrate ten of the most useful PlantUML
diagram types for technical and engineering documentation.\\
Each figure---such as the sequence diagram in \textbf{?@fig-seq} and the
deployment diagram in \textbf{?@fig-deploy}---is defined using a
\texttt{plantuml} code block and referenced in the surrounding text
using Quarto's cross-referencing system.

\bookmarksetup{startatroot}

\chapter{1. Sequence Diagram}\label{sequence-diagram}

Sequence diagrams capture \textbf{time-ordered interactions} between
participants.\\
In \textbf{?@fig-seq}, a user interacts with a web application and an
authentication service during login.

\begin{Shaded}
\begin{Highlighting}[]
\NormalTok{\#| label: fig{-}seq}
\NormalTok{\#| fig{-}cap: "Sequence Diagram for a User Login Flow"}
\NormalTok{@startuml}
\NormalTok{actor User}
\NormalTok{participant "Web App" as Web}
\NormalTok{participant "Auth Service" as Auth}

\NormalTok{User {-}\textgreater{} Web: Open login page}
\NormalTok{User {-}\textgreater{} Web: Submit credentials}
\NormalTok{Web  {-}\textgreater{} Auth: Verify(email, password)}
\NormalTok{Auth {-}{-}\textgreater{} Web: Token / Error}
\NormalTok{Web  {-}{-}\textgreater{} User: Dashboard / Error message}
\NormalTok{@enduml}
\end{Highlighting}
\end{Shaded}

\bookmarksetup{startatroot}

\chapter{2. Use Case Diagram}\label{use-case-diagram}

Use case diagrams show \textbf{system functionality as perceived by
actors}. In \textbf{?@fig-usecase}, we see a user interacting with an
authentication system.

\begin{Shaded}
\begin{Highlighting}[]
\NormalTok{\#| label: fig{-}usecase}
\NormalTok{\#| fig{-}cap: "Use Case Diagram for an Authentication System"}
\NormalTok{@startuml}
\NormalTok{left to right direction}

\NormalTok{actor User}

\NormalTok{rectangle "Authentication System" \{}
\NormalTok{  usecase "Log in" as UC1}
\NormalTok{  usecase "Reset password" as UC2}
\NormalTok{\}}

\NormalTok{User {-}{-}\textgreater{} UC1}
\NormalTok{User {-}{-}\textgreater{} UC2}
\NormalTok{@enduml}
\end{Highlighting}
\end{Shaded}

\bookmarksetup{startatroot}

\chapter{3. Class Diagram}\label{class-diagram}

Class diagrams depict \textbf{static structure}: classes, attributes,
operations, and relationships. Figure \textbf{?@fig-class} shows a
simple domain model for users, sessions, and an authentication service.

\begin{Shaded}
\begin{Highlighting}[]
\NormalTok{\#| label: fig{-}class}
\NormalTok{\#| fig{-}cap: "Class Diagram for a Simple Authentication Domain Model"}
\NormalTok{@startuml}
\NormalTok{class User \{}
\NormalTok{  +id: int}
\NormalTok{  +name: string}
\NormalTok{  +email: string}
\NormalTok{  +login()}
\NormalTok{\}}

\NormalTok{class Session \{}
\NormalTok{  +token: string}
\NormalTok{  +expiresAt: Date}
\NormalTok{\}}

\NormalTok{class AuthService \{}
\NormalTok{  +authenticate(email, password)}
\NormalTok{\}}

\NormalTok{User "1" {-}{-} "0..*" Session}
\NormalTok{AuthService ..\textgreater{} User}
\NormalTok{AuthService ..\textgreater{} Session}
\NormalTok{@enduml}
\end{Highlighting}
\end{Shaded}

\bookmarksetup{startatroot}

\chapter{4. Activity Diagram}\label{activity-diagram}

Activity diagrams show \textbf{workflows and control flow}. In
\textbf{?@fig-activity}, we model the steps of order processing,
including validation, inventory reservation, and payment.

\begin{Shaded}
\begin{Highlighting}[]
\NormalTok{\#| label: fig{-}activity}
\NormalTok{\#| fig{-}cap: "Activity Diagram for Order Processing"}
\NormalTok{@startuml}
\NormalTok{start}
\NormalTok{:Receive order;}
\NormalTok{:Validate order;}

\NormalTok{if (Valid?) then (yes)}
\NormalTok{  :Reserve inventory;}
\NormalTok{  :Process payment;}
\NormalTok{  if (Payment ok?) then (yes)}
\NormalTok{    :Confirm order;}
\NormalTok{  else (no)}
\NormalTok{    :Notify payment failure;}
\NormalTok{  endif}
\NormalTok{else (no)}
\NormalTok{  :Reject order;}
\NormalTok{endif}

\NormalTok{stop}
\NormalTok{@enduml}
\end{Highlighting}
\end{Shaded}

\bookmarksetup{startatroot}

\chapter{5. State Machine Diagram}\label{state-machine-diagram-1}

State machine diagrams represent \textbf{system states and transitions}.
In \textbf{?@fig-state}, an order moves from creation to delivery, with
rejection and completion paths.

\begin{Shaded}
\begin{Highlighting}[]
\NormalTok{\#| label: fig{-}state}
\NormalTok{\#| fig{-}cap: "State Machine Diagram for an Order Lifecycle"}
\NormalTok{@startuml}
\NormalTok{[*] {-}{-}\textgreater{} New}

\NormalTok{New {-}{-}\textgreater{} Approved : approve()}
\NormalTok{New {-}{-}\textgreater{} Rejected : reject()}

\NormalTok{Approved {-}{-}\textgreater{} Shipped : ship()}
\NormalTok{Shipped  {-}{-}\textgreater{} Delivered : deliver()}

\NormalTok{Rejected {-}{-}\textgreater{} [*]}
\NormalTok{Delivered {-}{-}\textgreater{} [*]}
\NormalTok{@enduml}
\end{Highlighting}
\end{Shaded}

\bookmarksetup{startatroot}

\chapter{6. Component Diagram}\label{component-diagram}

Component diagrams show \textbf{logical components} and their
interfaces. The web architecture in \textbf{?@fig-component} highlights
the separation between frontend, API gateway, user service, and the
database.

\begin{Shaded}
\begin{Highlighting}[]
\NormalTok{\#| label: fig{-}component}
\NormalTok{\#| fig{-}cap: "Component Diagram for a Web Application Architecture"}
\NormalTok{@startuml}
\NormalTok{component "Web Frontend" as FE}
\NormalTok{component "API Gateway"  as API}
\NormalTok{component "User Service" as US}
\NormalTok{database  "User DB"      as DB}

\NormalTok{FE  {-}{-}\textgreater{} API : HTTP/JSON}
\NormalTok{API {-}{-}\textgreater{} US  : REST}
\NormalTok{US  {-}{-}\textgreater{} DB  : SQL}
\NormalTok{@enduml}
\end{Highlighting}
\end{Shaded}

\bookmarksetup{startatroot}

\chapter{7. Deployment Diagram}\label{deployment-diagram}

Deployment diagrams focus on \textbf{where components run}. In
\textbf{?@fig-deploy}, we show a browser running on a user device,
connecting through the internet to a cloud-based web and API stack.

\begin{Shaded}
\begin{Highlighting}[]
\NormalTok{\#| label: fig{-}deploy}
\NormalTok{\#| fig{-}cap: "Deployment Diagram for a Cloud{-}Hosted Web System"}
\NormalTok{@startuml}
\NormalTok{node "User Device" \{}
\NormalTok{  artifact "Browser"}
\NormalTok{\}}

\NormalTok{node "Internet"}

\NormalTok{node "Cloud" \{}
\NormalTok{  node "Web Server" \{}
\NormalTok{    artifact "Web App"}
\NormalTok{  \}}
\NormalTok{  node "App Server" \{}
\NormalTok{    artifact "API Service"}
\NormalTok{  \}}
\NormalTok{\}}

\NormalTok{Browser {-}{-}\textgreater{} "Web App"}
\NormalTok{"Web App" {-}{-}\textgreater{} "API Service"}
\NormalTok{@enduml}
\end{Highlighting}
\end{Shaded}

\bookmarksetup{startatroot}

\chapter{8. Object Diagram}\label{object-diagram}

Object diagrams capture \textbf{snapshots of instances} at runtime.
Figure \textbf{?@fig-object} depicts one user, a shopping cart, and
multiple items.

\begin{Shaded}
\begin{Highlighting}[]
\NormalTok{\#| label: fig{-}object}
\NormalTok{\#| fig{-}cap: "Object Diagram Representing a Shopping Cart Instance"}
\NormalTok{@startuml}
\NormalTok{object user1 \{}
\NormalTok{  id   = 42}
\NormalTok{  name = "Alice"}
\NormalTok{\}}

\NormalTok{object cart1 \{}
\NormalTok{  total = 150.00}
\NormalTok{\}}

\NormalTok{object item1 \{}
\NormalTok{  sku = "B001"}
\NormalTok{  qty = 2}
\NormalTok{\}}

\NormalTok{object item2 \{}
\NormalTok{  sku = "C777"}
\NormalTok{  qty = 1}
\NormalTok{\}}

\NormalTok{user1 {-}{-} cart1}
\NormalTok{cart1 {-}{-} item1}
\NormalTok{cart1 {-}{-} item2}
\NormalTok{@enduml}
\end{Highlighting}
\end{Shaded}

\bookmarksetup{startatroot}

\chapter{9. Timing Diagram}\label{timing-diagram}

Timing diagrams show \textbf{state changes over time} for different
participants. In \textbf{?@fig-timing}, the client and server transition
through phases of a request--response cycle.

\begin{Shaded}
\begin{Highlighting}[]
\NormalTok{\#| label: fig{-}timing}
\NormalTok{\#| fig{-}cap: "Timing Diagram for a Simple Request–Response Protocol"}
\NormalTok{@startuml}
\NormalTok{robust "Client" as C}
\NormalTok{robust "Server" as S}

\NormalTok{@0}
\NormalTok{C is Idle}
\NormalTok{S is Idle}

\NormalTok{@10}
\NormalTok{C is Sending}

\NormalTok{@20}
\NormalTok{S is Processing}

\NormalTok{@30}
\NormalTok{C is Waiting}

\NormalTok{@40}
\NormalTok{S is Responding}

\NormalTok{@50}
\NormalTok{C is Idle}
\NormalTok{S is Idle}
\NormalTok{@enduml}
\end{Highlighting}
\end{Shaded}

\bookmarksetup{startatroot}

\chapter{10. Package Diagram}\label{package-diagram}

Package diagrams help \textbf{organize code into modules}. In
\textbf{?@fig-package}, we show a project divided into \texttt{Core} and
\texttt{UI} packages with their key classes and dependencies.

\begin{Shaded}
\begin{Highlighting}[]
\NormalTok{\#| label: fig{-}package}
\NormalTok{\#| fig{-}cap: "Package Diagram Organizing a Project into Core and UI Layers"}
\NormalTok{@startuml}
\NormalTok{package "Project" \{}
\NormalTok{  package "Core" \{}
\NormalTok{    class Engine}
\NormalTok{    class Model}
\NormalTok{  \}}
\NormalTok{  package "UI" \{}
\NormalTok{    class View}
\NormalTok{    class Controller}
\NormalTok{  \}}
\NormalTok{\}}

\NormalTok{Engine {-}{-}\textgreater{} Model}
\NormalTok{Controller {-}{-}\textgreater{} Engine}
\NormalTok{View {-}{-}\textgreater{} Controller}
\NormalTok{@enduml}
\end{Highlighting}
\end{Shaded}

\bookmarksetup{startatroot}

\chapter{Conclusion}\label{conclusion-2}

The ten PlantUML diagrams in this essay---sequence (\textbf{?@fig-seq}),
use case (\textbf{?@fig-usecase}), class (\textbf{?@fig-class}),
activity (\textbf{?@fig-activity}), state machine
(\textbf{?@fig-state}), component (\textbf{?@fig-component}), deployment
(\textbf{?@fig-deploy}), object (\textbf{?@fig-object}), timing
(\textbf{?@fig-timing}), and package (\textbf{?@fig-package})---cover a
broad spectrum of modeling needs.

Combining PlantUML with Quarto allows you to keep diagrams
\textbf{version-controlled, reproducible, and close to your prose},
making it ideal for teaching notes, software documentation, and research
reports.

\bookmarksetup{startatroot}

\chapter*{References}\label{references}
\addcontentsline{toc}{chapter}{References}

\markboth{References}{References}

\phantomsection\label{refs}
\begin{CSLReferences}{1}{0}
\bibitem[\citeproctext]{ref-knuth84}
Knuth, Donald E. 1984. {``Literate Programming.''} \emph{Comput. J.} 27
(2): 97--111. \url{https://doi.org/10.1093/comjnl/27.2.97}.

\end{CSLReferences}




\end{document}
